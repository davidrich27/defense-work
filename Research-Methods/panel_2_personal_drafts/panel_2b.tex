\documentclass{article}

\begin{document}
\begin{center}
Personal Statement\\
\end{center}
The same month that I graduated from high school, President Bill Clinton stood in the Rose Garden to announce the first ever sequencing of the human genome.  It seemed that both science and I were about to enter a new and exciting stage, but we were both in for a bit of a let down.  As a first generation college student, I found the whole experience daunting.  I flayed, picked a major on a whim, and graduated with no clear career path and with loan payments to make.   Meanwhile, the hype around the potential of genetic research to cure diseases like cancer soon faltered as it became apparent that it was all far more complicated than the talking heads had made it seem.  Scientists made their plodding progress and I was stuck waiting tables, knowing I could be doing more but without an idea of how.\\
\\
The Human Genome Project had taken 10 years and a few billion dollars to sequence just one genome, and the scientists still didn’t know what to make of all those As, Cs, Gs, and Ts.  Where was the Rosetta stone to translate this strange language?  Fortunately, engineers and computer scientists had ideas.  They built new sequencing technology and annotation software, and it all kept getting cheaper, faster, better.  I, on the other hand, was still feigning interest in a stranger’s choice of salad dressing.\\
\\
I finally decided that I would have to go back to school to find a career.  I strongly considered nursing but ultimately decided I don’t have the constitution for a frontline health care career (blood makes me queasy).  Next, I considered medical research, and everything I read said that to make it in modern biology you needed to know how to code, so I signed up for an introductory computer science class. \\
\\ 
Up until that point computer science had never crossed my mind as a career path.  I walked into that class having never written a single line of code, but I quickly discovered that not only was I good at it, I actually really enjoyed the work.  In my second semester of the introductory series my professor asked me about my interests.  I told him I wanted to work in genetics and he suggested I go talk to Dr. Travis Wheeler, a professor in the Computer Science program who did research in Computational Biology.  \\
\\
I began doing research with Dr. Wheeler and took his Computational Biology class. I was hooked.  The work was fun (I love puzzles), the problems were challenging (I had to understand both the biology and the algorithms) and the work was meaningful with potential applications to human health.  After completing some prerequisites I applied to the Master’s Program in Computer Science at the University of Montana.  I was accepted and offered a Research Assistant position with Dr. Wheeler.   \\
\\
When I began work on developing frameshift aware translated sequence alignment software my path once again came in contact with the path of genomic research.  This time it wasn’t just a coincidence, but an opportunity for me to make meaningful contribution to the field.   I began to see my role as a tool maker.  I might not be asking the big questions about the origins of life or the cures to human disease but I could do something just as important and, as it turns out, more suited to my talents.  I would build tools that made it possible to find the big answers.  I could help build the bridge between the hope and excitement in the Rose Garden all those years ago and reality of today’s scientific progress.  At the same time, I could build a smaller bridge, one just for me, between the hope and excitement I had felt when I graduated high school and the reality of having finally found my passion. \\

\end{document}



