\documentclass{article}

\title{Review of 2B}

\begin{document}
\maketitle

\section*{Rating:}
\begin{enumerate}
\item What is the potential for the proposed activity to:
  \begin{enumerate}
   \item Intellectual Merit? Excellent
   \item Broader Impacts? Excellent
  \end{enumerate}
\item Creative, original, or potentially transformative concepts? Very Good
\item Proposed activities well-reasoned, well-organized, and based on a sound rationale? Incorporate a mechanism to assess success? Excellent
\item How well qualified is the individual, team, or institution to conduct the proposed activities? Very Good
\item Adequate resources? Excellent
\end{enumerate}

\section*{Summary}
\subsection*{Intellectual Merit}
There seems to be a great deal of intellectual merit in this proposal.  This is a more robust version of this algorithm than has been implemented in the past, and the reasoning behind the algorithm seems grounded in good science.  It is plain to see that it will be more sensitive than previous alignment methods and this will meaningfully contribute to the research in the field.

\subsection*{Broader Impacts}
The broader impacts are mostly apparent in this proposal.  As this more alignment algorithm searches in a space currently unexamined, these spaces will reveal as yet unannoted sequences as well as help with acquiring alignments even in the presence read-errors and indels.  This has high likelihood to progress many types of genetic research.

\section*{Summary Statement}
Everything in the proposal seems to be very good to excellent in quality.  Only issues is to perhaps expand the method/assessment section, maybe making more explicit how the accuracy of the software will be analyzed.  With respect to being well-qualified and having adequate resources, not much is mentioned.  But qualifications might fit better in the personal statement.  With respect to having adequate resources, maybe have a brief statement about HMMER being free and open source, as well as having access to the necessary genetic databases for testing purposes.
\end{document}
