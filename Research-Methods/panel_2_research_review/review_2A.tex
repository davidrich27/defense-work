\documentclass{article}

\title{Review of 2A}

\begin{document}
\maketitle

\section*{Rating:}
\begin{enumerate}
\item What is the potential for the proposed activity to:
  \begin{enumerate}
   \item Intellectual Merit? Very Good
   \item Broader Impacts? Very Good
  \end{enumerate}
\item Creative, original, or potentially transformative concepts? Good
\item Proposed activities well-reasoned, well-organized, and based on a sound rationale? Incorporate a mechanism to assess success? Good
\item How well qualified is the individual, team, or institution to conduct the proposed activities? Very Good
\item Adequate resources? Very Good
\end{enumerate}

\section*{Summary}
\subsection*{Intellectual Merit}
The intellectual merit seems is quite in this proposal.   Establishing ground truth seems to be critical for the analysis of algorithms in the field of mass spectrometry.  

\subsection*{Broader Impacts}
The broader impacts are mostly apparent in this proposal.  The user-friendly data visualization JS-MS software is great for lowering the technological barriers and increasing the effectiveness of contributions from non-computer science community into mass spectrometry.  

\section*{Summary Statement}
Everything in the proposal seems to be very good in quality.  The work on JS-MS seems good.  I think what lacked the most is forms of assessment.  There didn't seem to be much for verifying results.  For JS-MS, maybe have usability tests.  Or if there is another similar software suite for MS, maybe comparing against that.  And for ground truth data, it feels like there should be some way to show that personally-annotated MS data is superior to algorithmic annotation and that it is consistent.  Final note, the proposal is written from the prospective that the work is already completed.
\end{document} 
