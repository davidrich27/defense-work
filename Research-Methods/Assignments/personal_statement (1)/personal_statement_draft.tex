  
\documentclass{article}
\usepackage[T1]{fontenc}
\usepackage[utf8]{inputenc}
\usepackage[margin=1.75in]{geometry}


\newcommand{\HRule}{\rule{\linewidth}{0.5mm}}
\newcommand{\Hrule}{\rule{\linewidth}{0.3mm}}

\makeatletter% since there's an at-sign (@) in the command name
\renewcommand{\@maketitle}{%
	\parindent=0pt% don't indent paragraphs in the title block
	\centering
	{\Large \bfseries\textsc{\@title}}
	\HRule\par%
	\textit{\@author \hfill \@date}
	\par
}
\makeatother% resets the meaning of the at-sign (@)

\title{Personal Statement Draft}
\author{David H. Rich}
\date{Grant Applicant}

\begin{document}
	\maketitle% prints the title block
	\thispagestyle{empty}
	\vspace{16pt}
	
	\section{Early Experiences}
	
	When I graduated high school, I had no idea I would eventually go into a science or technology field.  While I did well enough in my math and science courses, I went to a small town high school with no computer science program (other than a self-taught HTML course), and the minimal budget meant there was very little opportunity for any hands-on science labs. As a result, it was difficult to cultivate much interest in these subjects.  At that point in my life, I was more interested in creative pursuits like art.  
	
	I graduated high school in 2007.  Neither of my parents went to college, but they strongly encouraged me and my sister to go.  I decided on attending the local University of Montana for my undergraduate studies.
	
	When I went to college, I was unsure of what I wanted to do and I spent my first two or so years bouncing between various majors including art, math and economics.  Well into my college career, I finally landed on business.  During that time, I also began taking information systems and computer science courses.  
	
	These programming courses were completely revelatory when compared to the introductory web development courses I had taken in the past.  I loved how the problem-solving in computer science involved both critical thinking and creativity.  I will always remember the sense of pride I got when I built my first program from scratch in my programming intro course; it was a Mario Bros clone in Alice.  Or when I finally got the hang of using pointers in my first C/C++ course.  
	
	  However, I was near graduation by the time I began my first computer science courses.  Rather than change tracks and extend my college career, I completed with my BS in business in 2012. 
	
	From high school up through college, I'd spent my summers working at the my parent’s business, who own and operate their own guest ranch.  During the summers I would pack mules and take guests into the Bob Marshall Wilderness.  It was a great way to spend my summers and gave me the opportunity to meet people from all walks of life.  After school, I fought fire for two summers and went back to working at the ranch full-time, as well as managing the ranch website for them.  While working there, I continued learning programming online during my free time, and in the back of my mind I knew I would eventually return to school.
	
	I returned to the University of Montana in 2017.  Originally, I planned to only get my post-bacc in computer science and mathematics.  But once I was about a year into my studies, I was approached by Travis Wheeler with a teaching assistant position, so I entered the Master's program. 
	
	\pagebreak
	\section{Career Goals and Current Research}
	
	A year later after entering the Master’s program, Travis Wheeler offered me a research position in his bioinformatics lab.  While I was initially nervous, as I had very little knowledge of genomics going in, this year has been an exciting dive into the world of biology and scientific computing.  My time here has given me a real appreciation for the research process.  My current research is in implementing a heuristic version of the forward/backward algorithm for the HMMER suite pipeline.  We hope that this will cause a sizable speedup to HMMER's runtime at minimal cost to accuracy.
	
	The anticipated completion date for my Master’s work is early spring 2020.  As for the future, I have no current plans to enter any PhD program.  This most likely means work in industry for a private company.  However, if the opportunity arises, I would be interested in continuing to work in scientific computing,  whether that means in bioinformatics or another natural science field.  I am also intrigued by work in environmental sciences, as I have spent much of my life out in the wilderness and would love to do work that could benefit the natural world.
	
	Ultimately, I enjoy scientific computing because they are always asking you to learn something new, and introducing new puzzles that require novel solutions.  
	
\end{document}



